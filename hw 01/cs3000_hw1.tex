\documentclass[11pt]{article}

\newcommand{\yourname}{}
\newcommand{\yourcollaborators}{}

\def\comments{0}

%format and packages

%\usepackage{algorithm, algorithmic}
\usepackage{algpseudocode}
\usepackage{amsmath, amssymb, amsthm}
\usepackage{enumerate}
\usepackage{enumitem}
\usepackage{framed}
\usepackage{verbatim}
\usepackage[margin=1.0in]{geometry}
\usepackage{microtype}
\usepackage{kpfonts}
\usepackage{palatino}
	\DeclareMathAlphabet{\mathtt}{OT1}{cmtt}{m}{n}
	\SetMathAlphabet{\mathtt}{bold}{OT1}{cmtt}{bx}{n}
	\DeclareMathAlphabet{\mathsf}{OT1}{cmss}{m}{n}
	\SetMathAlphabet{\mathsf}{bold}{OT1}{cmss}{bx}{n}
	\renewcommand*\ttdefault{cmtt}
	\renewcommand*\sfdefault{cmss}
	\renewcommand{\baselinestretch}{1.06}
\usepackage[usenames,dvipsnames]{xcolor}
\definecolor{DarkGreen}{rgb}{0.15,0.5,0.15}
\definecolor{DarkRed}{rgb}{0.6,0.2,0.2}
\definecolor{DarkBlue}{rgb}{0.2,0.2,0.6}
\definecolor{DarkPurple}{rgb}{0.4,0.2,0.4}
\usepackage[pdftex]{hyperref}
\hypersetup{
	linktocpage=true,
	colorlinks=true,				% false: boxed links; true: colored links
	linkcolor=DarkBlue,		% color of internal links
	citecolor=DarkBlue,	% color of links to bibliography
	urlcolor=DarkBlue,		% color of external links
}

\usepackage[boxruled,vlined,nofillcomment]{algorithm2e}
	\SetKwProg{Fn}{Function}{\string:}{}
	\SetKwFor{While}{While}{}{}
	\SetKwFor{For}{For}{}{}
	\SetKwIF{If}{ElseIf}{Else}{If}{:}{ElseIf}{Else}{:}
	\SetKw{Return}{Return}
	

%enclosure macros
\newcommand{\paren}[1]{\ensuremath{\left( {#1} \right)}}
\newcommand{\bracket}[1]{\ensuremath{\left\{ {#1} \right\}}}
\renewcommand{\sb}[1]{\ensuremath{\left[ {#1} \right\]}}
\newcommand{\ab}[1]{\ensuremath{\left\langle {#1} \right\rangle}}

%probability macros
\newcommand{\ex}[2]{{\ifx&#1& \mathbb{E} \else \underset{#1}{\mathbb{E}} \fi \left[#2\right]}}
\newcommand{\pr}[2]{{\ifx&#1& \mathbb{P} \else \underset{#1}{\mathbb{P}} \fi \left[#2\right]}}
\newcommand{\var}[2]{{\ifx&#1& \mathrm{Var} \else \underset{#1}{\mathrm{Var}} \fi \left[#2\right]}}

%useful CS macros
\newcommand{\poly}{\mathrm{poly}}
\newcommand{\polylog}{\mathrm{polylog}}
\newcommand{\zo}{\{0,1\}}
\newcommand{\pmo}{\{\pm1\}}
\newcommand{\getsr}{\gets_{\mbox{\tiny R}}}
\newcommand{\card}[1]{\left| #1 \right|}
\newcommand{\set}[1]{\left\{#1\right\}}
\newcommand{\negl}{\mathrm{negl}}
\newcommand{\eps}{\varepsilon}
\DeclareMathOperator*{\argmin}{arg\,min}
\DeclareMathOperator*{\argmax}{arg\,max}
\newcommand{\eqand}{\qquad \textrm{and} \qquad}
\newcommand{\ind}[1]{\mathbb{I}\{#1\}}
\newcommand{\sslash}{\ensuremath{\mathbin{/\mkern-3mu/}}}

%mathbb
\newcommand{\N}{\mathbb{N}}
\newcommand{\R}{\mathbb{R}}
\newcommand{\Z}{\mathbb{Z}}
%mathcal
\newcommand{\cA}{\mathcal{A}}
\newcommand{\cB}{\mathcal{B}}
\newcommand{\cC}{\mathcal{C}}
\newcommand{\cD}{\mathcal{D}}
\newcommand{\cE}{\mathcal{E}}
\newcommand{\cF}{\mathcal{F}}
\newcommand{\cL}{\mathcal{L}}
\newcommand{\cM}{\mathcal{M}}
\newcommand{\cO}{\mathcal{O}}
\newcommand{\cP}{\mathcal{P}}
\newcommand{\cQ}{\mathcal{Q}}
\newcommand{\cR}{\mathcal{R}}
\newcommand{\cS}{\mathcal{S}}
\newcommand{\cU}{\mathcal{U}}
\newcommand{\cV}{\mathcal{V}}
\newcommand{\cW}{\mathcal{W}}
\newcommand{\cX}{\mathcal{X}}
\newcommand{\cY}{\mathcal{Y}}
\newcommand{\cZ}{\mathcal{Z}}

%theorem macros
\newtheorem{thm}{Theorem}
\newtheorem{lem}[thm]{Lemma}
\newtheorem{fact}[thm]{Fact}
\newtheorem{clm}[thm]{Claim}
\newtheorem{rem}[thm]{Remark}
\newtheorem{coro}[thm]{Corollary}
\newtheorem{prop}[thm]{Proposition}
\newtheorem{conj}[thm]{Conjecture}

\theoremstyle{definition}
\newtheorem{defn}[thm]{Definition}


\newcommand{\instructor}{Drew van der Poel}
\newcommand{\hwnum}{1}
\newcommand{\hwdue}{Tuesday, January 21 at 11:59pm via \href{https://gradescope.com/courses/74204}{Gradescope}}

\theoremstyle{theorem}
\newtheorem{prob}{Problem}
\newtheorem{sol}{Solution}

\definecolor{cit}{rgb}{0.05,0.2,0.45} 
\newcommand{\solution}{\medskip\noindent{\color{DarkBlue}\textbf{Solution:}}}

\begin{document}
{\Large 
\begin{center}{CS3000: Algorithms \& Data} --- Spring '20 --- \instructor \end{center}}
{\large
\vspace{10pt}
\noindent Homework~\hwnum \vspace{2pt}\\
Due~\hwdue}

\bigskip
{\large
\noindent Name: \yourname \vspace{2pt}\\ Collaborators: \yourcollaborators}

\vspace{15pt}
\begin{itemize}

\item Make sure to put your name on the first page.  If you are using the \LaTeX~template we provided, then you can make sure it appears by filling in the \texttt{yourname} command.

\item This assignment is due~\hwdue.  No late assignments will be accepted.  Make sure to submit something before the deadline.

\item Solutions must be typeset in \LaTeX.  If you need to draw any diagrams, you may draw them by hand as long as they are embedded in the PDF.  I recommend using the source file for this assignment to get started.

\item I encourage you to work with your classmates on the homework problems. \emph{If you do collaborate, you must write all solutions by yourself, in your own words.}  Do not submit anything you cannot explain.  Please list all your collaborators in your solution for each problem by filling in the \texttt{yourcollaborators} command.

\item Finding solutions to homework problems on the web, or by asking students not enrolled in the class is strictly forbidden.

\end{itemize}

\newpage

\begin{prob} Inductive Proofs (20 points) \end{prob}
\begin{enumerate}[label=(\alph*)]
\item \textbf{[8 points]} Prove the following statement by induction: For every $n \in \N$, $\sum_{i=1}^{n} i^2 = \frac{n(n+1)(2n+1)}{6}$

\solution
\\Let's consider the base case, where n = 1:
\begin{equation*}
	1^2 = \frac{1 \cdot 2 \cdot 3} {6}
\end{equation*}
This is true.
\\Assuming this is true for n = k - 1 
\begin{equation*}
	1^2 + 2^2 + 3^2 + \ldots + (k - 1)^2 = \frac{(k-1) \cdot (k) \cdot (2(k - 1)+1)} {6}  \tag{1}
\end{equation*}

Inductive case:
we can prove that this is also true for n = k

\begin{equation*}
	1^2 + 2^2 + 3^2 + \ldots + (k - 1)^2 + k^2 
\end{equation*}

Substituting (1) in (2)
\begin{equation*}
	\frac{(k-1) \cdot (k) \cdot (2(k - 1)+1)} {6}+ (k+1)^2 
\end{equation*}

this can be written as:

\begin{align*}
	(1^2 + 2^2 + 3^2 + \ldots + (k-1)^2 ) + (k)^2 &=\frac{(k-1) \cdot k \cdot (2(k - 1)+1)} {6} + k^2 \\
									&=k\cdot (\frac{(k-1)\cdot(2k-1)}{6} + k)\\
									&=k\cdot (\frac{2k^2 +3k + 1}{6})\\
									&=k\cdot (\frac{2k^2 +2k + k + 1}{6})\\
									&= \frac{(k)(k+1)(2k+1)}{6}
\end{align*}

\qed

\item \textbf{[8 points]} Prove the following statement by induction: For every $n \in \N$, $\sum_{i=1}^{n} \frac{1}{i^2} \leq 2 - \frac{1}{n}$

\solution
\\Let's consider the base case, n = 1:


\item \textbf{[4 points]} Your friend shows you the following dubious theorem and proof.

\begin{thm} In every set of $n \geq 1$ dice, all dice are the same color.   \end{thm}
\begin{proof}

\textbf{Inductive Hypothesis:} Let $H(k)$ be the statement: in every set of $k$ dice, all of the $k$ dice are the same color.  We will prove that $H(k)$ is true for every $k \in \N$.

\noindent \textbf{Base Case:} Consider $H(1)$.  Because the set has only one die, it is the same color at itself, so $H(1)$ is true.

\noindent \textbf{Inductive Step:}  We will show that for every $k \geq 1$, $H(k) \Longrightarrow H(k+1)$. Assume that $H(k)$ is true.  Consider a set of $k+1$ dice $d_1, \ldots, d_k, d_{k+1}$. By our assumption, the first $k$ dice are the same color.
$$\underbrace{d_1, d_2, \ldots, d_k}_\text{same color}, d_{k+1}$$
Also by our assumption, the last $k$ dice also have the same color.
$$d_1,\underbrace{d_2, \ldots, d_k, d_{k+1}}_\text{same color}$$
Therefore, by transitivity, all dice are the same color.

Therefore, the claim holds for all $n$ by induction.
\end{proof}

What is the bug in this proof?

\solution

\end{enumerate}

\newpage
\begin{prob} Stable Matching (20 points) \end{prob}
In class we showed that given \emph{any} set of rankings for $n$ doctors and $n$ hospitals, there always exists at least one stable matching of doctors and hospitals.  

\begin{enumerate}[label=(\alph*)]
\item  \textbf{[5 points]} Show that there is a set of rankings for $2$ doctors and $2$ hospitals such that there is a \emph{unique} stable matching.  Justify the claim that there is a unique stable matching.

\solution

\item  \textbf{[5 points]} Show that there is a set of rankings for $2$ doctors and $2$ hospitals such that there exist two distinct stable matchings.

\solution

\item  \textbf{[10 points]} Show that, for every $n$, there is a set of rankings for $2n$ doctors and $2n$ hospitals such that there are at least $2^n$ distinct stable matchings.  \emph{Hint: start with your answer to part (b) and build your ranking two pairs at a time.}

\solution

\end{enumerate}

\newpage
\begin{prob} Asymptotic Order of Growth (20 points) \end{prob}
\begin{enumerate}[label=(\alph*)]
\item  \textbf{[10 points]} Rank the following functions in increasing order of asymptotic growth rate.  That is, find an ordering $f_1, f_2, \ldots, f_{10}$ of the functions so that $f_i = O(f_{i+1})$. No justification is required.

\begin{center}
\begin{tabular}{ccccc}
$n^3$ & $7^{\log_2 n}$ & $n!$ & $12^n$ & $\log_2 (n!)$  \\
$2^{4n}$ & $100 n^{3/2}$ & $10n$ & $2^{\log_3 n}$ & $\log_2^3 n$
\end{tabular}
\end{center}

\solution

\vspace{-20pt}
\begin{align*}
    f_1(n) ={} &??? \\
    f_2(n) ={} &??? \\
    \dots &
\end{align*}

\item  \textbf{[10 points]} Suppose $f(n), g(n), h(n)$ are non-decreasing, non-negative functions and that $f(n) = O(h(n))$ and $g(n) = O(h(n))$.  Prove that the $f(n)g(n) = O(h(n)^2)$.

\solution

\end{enumerate}

\newpage
\begin{prob} What Does This Code Do? (20 points)\end{prob}

You encounter the following mysterious piece of code.

\begin{algorithm}
\caption{Mystery Function}
\Fn{$C(a,n)$}{
	\If{$n=0$}{\Return $(1,a)$}
	\ElseIf{$n=1$}{\Return $(a,a\cdot a)$}
	\ElseIf{$n$ is even}{
		$(u,v) \gets C(a, \lfloor n/2 \rfloor)$\\
	\Return $(u \cdot u, u \cdot v)$
	}\ElseIf{$n$ is odd}{
		$(u,v) \gets C(a, \lfloor n/2 \rfloor)$\\
		\Return $(u \cdot v, v \cdot v)$
	}
}
\end{algorithm}

\begin{enumerate}[label=(\alph*)]
\item  \textbf{[3 points]}  What are the results of $C(a,2)$, $C(a,3)$, and $C(a,4)$.  You do not need to justify your answers.

\solution

\item   \textbf{[9 points]} What does the code do in general? Prove your assertion by induction on $n$.

\solution

\item  \textbf{[8 points]} In this problem you will analyze the running time of $C$ as a function of $n$.  Prove that, for every $n \in \N$, the number of multiplication operations performed in evaluating $C(a,n)$ is at most $2 \log_2 n + 1$.

\solution

\end{enumerate}

\end{document}
